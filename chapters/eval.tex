\chapter{\ifproject%
\ifenglish Experimentation and Results\else การทดลองและผลลัพธ์\fi
\else%
\ifenglish System Evaluation\else การประเมินระบบ\fi
\fi}

% \section{การประเมินระบบ}
\enskip \enskip \enskip \enskip \enskip ในบทนี้จะเป็นการทดสอบระบบของตัวเกมว่ามีความสมบูรณ์มากน้อยเพียงใด โดยแบ่งหมวดหมู่เป็น ความสมบูรณ์ของตัวละคร  ความสมบูรณ์ของศัตรู  แผนที่ การบรรลุเป้าหมายการเล่น และความสวยงามของเกม โดยรายละเอียดของแต่ละหมวดหมู่นั้นอธิบายได้ดังนี้
\section{การทดสอบความสมบูรณ์ของตัวละคร}
การทดสอบระบบการควบคุมตัวละคร จะทดสอบพื้นฐานคือการเดินหน้า ถอยหลัง ซ้าย ขวา กระโดด การเปลี่ยนอาวุธ การใช้Active SkillและUltimate Skill การหมุนมุมกล้องของตัวละครในแต่ละพื้นที่ของแมพ ไม่ให้มุมกล้องหลุดไปจากตัวละคร หรือมีบางส่วนของแผนที่บังมุมมองของผู้เล่น 
\section{ความสมบูรณ์ของศัตรู }
ศัตรูในเกมจะต้องไม่ยากเกินไปจนทำให้ผู้เล่นไม่สามารถรับมือได้ หรือมีความบกพร่องที่ไม่สามารถทำให้ไปต่อได้ และไม่ง่ายเกินจนทำให้เกมน่าเบื่อไม่มีความท้าทาย

\section{ความสมบูรณ์ของแผนที่}
การหาจุดบกพร่องของแผนที่ในจุดต่างๆภายในเกม ไม่ให้มีจุดที่ตัวละครเดินไปแล้วเกิดหลุดออกจากแผนที่ ไม่สามารถเดินไปในที่ที่ควรเดินไปได้ หรือเดินไปแล้วตัวละครติดในพื้นที่นั้นไม่สามารถนำตัวละครออกมาได้ อีกทั้งด้านความสวยงามของแผนที่ว่ามีอะไรที่แสดงผลออกมาไม่เหมือนที่ต้องการหรือไม่ จะทำการตรวจสอบโดยการนำตัวละครไปเดินในแต่ละจุด ที่คิดว่าอาจเกิดข้อบกพร่อง 

\section{การบรรลุเป้าหมายในการเล่น}
ผู้เล่นต้องสามารถทำเควสหลักและรองได้โดยไม่เกิดปัญหาสามารถเดินทางไปยังพื้นที่ต่างๆ เมื่อจัดการบอสของพื้นที่แล้วต้องได้รับไอเทมที่นำไปใช้สู้กับจอมมาร และเมื่อชนะจอมมารเกมต้องจบลงและมีการกลับไปหน้าหลักเพื่อให้ผู้เล่นสามารถเริ่มเล่นใหม่อีกกี่รอบก็ได้เมื่อจบเกม

\section{ความสวยงามของตัวเกม}
ด้านความสวยงามก็เป็นส่วนสำคัญที่จะทำให้ตัวเกมดูน่าสนใจและน่าเล่นมากยิ่งขึ้น ไม่เพียงแค่ความ สวยงามภายในระบบเกมที่เราเล่น แต่รวมไปถึงหน้า UI ก็ส่งผลให้ผู้เล่นสนใจเล่นเกมของเรา ความสวยงามของสิ่งแวดล้อมในเกม เพลงประกอบที่ให้อารมณ์ความรู้สึกร่วม ตัวละครหลักที่มีความน่ารักสดใส มอนสเตอร์ที่มีความน่าเกรงขาม องค์ประกอบเหล่านี้จะทำให้ผู้เล่นสามารถสนุกไปกับเกมของเราได้มากขึ้น

