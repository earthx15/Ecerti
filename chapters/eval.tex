\chapter{\ifproject%
\ifenglish Experimentation and Results\else การทดลองและผลลัพธ์\fi
\else%
\ifenglish System Evaluation\else การประเมินระบบ\fi
\fi}

% \section{การประเมินระบบ}
\enskip \enskip \enskip \enskip \enskip เนื้อหาในบทนี้จะเกี่ยวกับการในการทดสอบการทำงานของโปรแกรม และการ ประเมินความพึงพอใจจากผู้ ใช้งาน โดยการสัมภาษณ์ผู้มีประสบการณ์ในการใช้งานเครื่องมือที่มีความใกล้เคียง เพื่อประเมินประสิทธิภาพ ของโปรแกรมและนำข้อมูลที่ได้มาใช้ในการปรับปรุงแก้ไขให้ระบบสามารถทำงานได้ตาม แผนที่วางไว้และตรง ความต้องการของผู้ใช้มากขึ้น
\section{การทดสอบระบบ}
การทดสอบการทำงานของระบบในส่วนนี้ทดสอบโดยผู้จัดทำาเพื่อทดสอบว่าฟังก์ชันหลักแต่ละฟังก์ชันสามารถ ทำงานได้ถูกต้องตามผลลัพธ์ที่วางแผนไว้ โดยมีหัวข้อในการทดสอบแบ่งตามกลุ่มผู้ใช้งานดังนี้
\subsection{แอพพลิเคชั่นฝั่งผู้ออกใบประกาศนียบัตร}
\begin{enumerate}
\item การเข้าสู่ระบบสามารถทำงานได้ถูกต้อง
\item การสมัครสมาชิกยังคงมีปัญหาในกรณี username ซ้ำ
\item ระบบ OTP ทำงานได้ถูกต้องแต่ยังไม่สามารถกำหนดระยะเวลาของ OTP
\item ผู้ใช้สามารถออกใบประกาศนียบัตรอย่างรวดเร็วได้ถูกต้อง
\item ผู้ใช้สามารถสร้างใบประกาศนียบัตรสำหรับหนึ่งคนได้ถูกต้อง
\item ผู้ใช้สามาถสร้างคอร์สใหม่ได้ถูกต้อง
\item ผู้ใช้สามารถตรวจสอบข้อมูลแต่ละคอร์สได้ถูกต้อง
\item ผู้ใช้สามารถออกใบประกาศนียบัตรสำหรับหลายๆคนในหนึ่งคอร์สได้ถูกต้องแต่ยังใช้เวลานานในการออก
\item ผู้ใช้สามารถ Export ไฟล์ Excel ได้ถูกต้อง
\item ผู้ใช้สามารถ Import ไฟล์ Excel ได้ถูกต้อง
\item ระบบส่งใบประกาศนียบัตรทางอีเมล์ให้นักเรียนในคอร์สทำงานได้ถูกต้อง
\item ระบบได้ทำการบันทึกข้อมูลของใบประกาศนียบัตรลงใน Blockchain ได้ถูกต้อง
\end{enumerate}
\subsection{แอพพลิเคชั่นฝั่งผู้ตรวจใบประกาศนียบัตร}
\begin{enumerate}
    \item ผู้ใช้สามารถแสกน QR Code ได้ถูกต้อง
    \item ระบบโชว์ใบประกาศนียบัตรได้ถูกต้อง
\end{enumerate}
จากการทดสอบในหัวข้อของการทดสอบการทำงานของระบบโดยแบ่งตามผู้ใช้กลุ่มต่างๆพบว่าแต่ละฟังก์ชันของแอพพลิเคชั่นสามารถทำงานและแสดงผลลัพธ์ได้อย่างถูกต้องตามฟังก์ชันที่ต้องการใช้งาน
