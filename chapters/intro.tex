\chapter{\ifenglish Introduction\else บทนำ\fi}

\section{\ifenglish Project rationale\else ที่มาของโครงงาน\fi}
% \tolerance = 9999
% \overfullrule=0pt
\enskip \enskip \enskip \enskip \enskip 
เกมเป็นสื่อบันเทิงประเภทหนึ่งที่มีการแพร่หลายเป็นอย่างมากในปัจจุบัน ไม่ว่าจะเป็นเกมบนมือถือ บนเว็บไซต์ เกมบนเครื่องเล่นเกมต่างๆที่ออกแบบมาเพื่อเกมใดเกมหนึ่งโดยเฉพาะ และรวมไปถึงเกมบนคอมพิวเตอร์ ซึ่งบางเกมได้มีการจัดการแข่งขันกันขึ้น เพื่อชิงรางวัลต่างๆมากมายภายในงานเเข่ง ส่งผลให้ผู้คนให้ความสนใจกับเกมมากขึ้น และส่งผลให้อุสาหกรรมเกมมีการเติบโตอย่างรวดเร็ว จนเกิดอาชีพใหม่ๆมากมายที่เกี่ยวกับเกม เช่น Streamer, นักกีฬา E-sport, นักพากย์เกม เป็นต้น   

\enskip \enskip
โดยโครงงานนี้ก็ได้เริ่มต้นมาจากการที่ผู้พัฒนาชื่นชอบในการเล่นเกม และมีความสนใจที่จะสร้างเกมขึ้นมาหนึ่งเกม ซึ่งผู้พัฒนาได้ลองทำการศึกษาพื้นฐานต่างๆเกี่ยวกับการสร้างเกม เเละตัดสินใจเสนอความสนใจเหล่านี้พร้อมกับอธิบายเหตุผลให้กับอาจารย์ฟัง จนสุดท้ายได้ทำการตกลงกับอาจารย์ว่าจะสร้างเกม 3D แนว RPG action ขึ้นมา ซึ่งก็คือ เกม Miracle from sky นั่นเอง 

\enskip \enskip สำหรับเกม Miracle from sky เป็นเกมเเนว Action RPG OpenWorld แบบ Single-player ที่มีมุมมองเป็น มุมมองของบุคคลที่สาม ซึ่งผู้พัฒนาให้ความสนใจ และอยากนำมาเป็นต้นแบบในการทำเกมคือ Genshin impact และ Diablo ซึ่งทางระบบ gameplay จะเน้นไปทาง Genshin impact ส่วนระบบสกิลจะเน้นไปทาง Diablo ซึ่งในเกม ผู้เล่นจะได้รับบทเป็นเด็กสาวที่ต้องผจญภัยในโลกกว้าง และฝึกฝนตัวเองให้เก่งขึ้น เพื่อที่จะไปปราบจอมมาร โดยระหว่างการเดินทางผู้เล่นจะได้พบศัตรูหลากหลายรูปแบบ ซึ่งต้องใช้วิธีรับมือที่แตกต่างกัน ได้สำรวจโลกแฟนตาซีกว้างใหญ่ และได้พบเจอกับปริศนาต่างๆที่รอให้ผู้เล่นได้เข้าไปแก้ไขหาคำตอบ 
\section{\ifenglish Objectives\else วัตถุประสงค์ของโครงงาน\fi}
\begin{enumerate}
    \item เพื่อตอบสนองความสนใจ ความต้องการของผู้พัฒนาที่อยากจะทำเกมของตัวเองขึ้นมาซักหนึ่งเกม
    \item เพื่อสร้างประสบการณ์ต่างๆที่น่าตื่นเต้น สนุกสนาน และน่าติดตามให้กับผู้เล่น ผ่านทางตัวเกม ทั้งด้านเนื้อเรื่อง gameplay และสิ่งต่างๆภายในเกม
    \item เพื่อสร้างความบันเทิงให้กับผู้เล่น และช่วยทำให้ผู้เล่นรู้สึกผ่อนคลายเวลาเล่นเกม
    \item เพื่อเป็นแบบอย่าง และแรงบรรดาลใจให้กับหลายๆคนที่อยากจะลองสร้างเกมของตัวเองขึ้นมา
\end{enumerate}

\section{\ifenglish Project scope\else ขอบเขตของโครงงาน\fi}

\subsection{\ifenglish Hardware scope\else ขอบเขตด้านฮาร์ดแวร์\fi}
\begin{enumerate}
    \item เกมสามารถเล่นได้ผ่านทางคอมพิวเตอร์ ซึ่งได้แก่ PC, laptop
    \item เกมจะใช้เมาส์ แป้นพิมพ์ ในการควบคุม
\end{enumerate}
\subsection{\ifenglish Software scope\else ขอบเขตด้านซอฟต์แวร์\fi}
\begin{enumerate}
    \item เกมจะรองรับแค่ระบบปฏิบัติการ Windows
    \item เกมถูกออกแบบมาสำหรับผู้เล่นคนเดียว
    \item เกมมีมุมมองเป็นเเบบมุมมองบุคคลที่สาม เท่านั้น ไม่สามารถเปลี่ยนมุมมองอื่นๆได้
\end{enumerate}
\section{\ifenglish Expected outcomes\else ประโยชน์ที่ได้รับ\fi}
\begin{enumerate}
    \item ผู้เล่นจะได้รับความสนุกสนาน ความบันเทิงต่างๆภายในเกม
    \item ผู้เล่นจะได้รับประสบการณ์ใหม่ๆมากมายจากการเกม
    \item ผู้พัฒนาได้รับประสบการณ์ใหม่ๆในการทำงานเป็นทีม และประสบการณ์ต่างๆในการสร้างเกม ซึ่งเป็นผลดีต่อการทำงานในอนาคต
\end{enumerate}
\section{\ifenglish Technology and tools\else เทคโนโลยีและเครื่องมือที่ใช้\fi}

\subsection{\ifenglish Hardware technology\else เทคโนโลยีด้านฮาร์ดแวร์\fi}
\begin{enumerate}
    \item คอมพิวเตอร์รุ่น Asus zephyrus g14, Ryzen 7 ใช้ในการออกแบบ และพัฒนาเกม
    \item คอมพิวเตอร์รุ่น Acer aspire5, core i7 ใช้ในการออกแบบ และพัฒนาเกม โดยจะใช้คอมเครื่องนี้เป็นตัวหลักในการสร้างโปรเจคหลัก
\end{enumerate}
\subsection{\ifenglish Software technology\else เทคโนโลยีด้านซอฟต์แวร์\fi}
\begin{enumerate}
    \item ระบบปฏิบัติการ Windows โดยจะใช้เป็น Window 10
    \item Unity ใช้เป็นตัวหลักในการพัฒนาเกม โดยจะใช้ platform 3D ของ Unity ในการสร้างเกม
    % \item Visual Studio โดยใช้ในการเขียน ในการควบระบบของเกม ซึ่งใช้ภาษา ในการเขียน code
    \item Visual studio ใช้ในการเขียน script ควบคุมระบบเกมต่างๆ ซึ่งใช้ภาษา \texttt{C\#} ในการเขียน code
    \item Blender ใช้ในการสร้างโมเดล 3D สำหรับใช้ในเกม 
    \item Krita ใช้ในการออกแบบและ ช่วยในการสร้างเอฟเฟคต่างๆในเกม 
    \item Photoshop ใช้ในการออกแบบและ ช่วยในการสร้างเอฟเฟคต่างๆในเกม 
\end{enumerate}
\section{\ifenglish Project plan\else แผนการดำเนินงาน\fi}

\begin{plan}{8}{2023}{3}{2024}
    \planitem{8}{2023}{9}{2023}{ศึกษาค้นคว้าการใช้งาน Unity}
    \planitem{9}{2023}{10}{2023}{ศึกษาค้นคว้าการใช้ Visual studio ในการเขียน script}
    \planitem{9}{2023}{11}{2023}{วางแผนออกแบบเกม เช่น ออกเเบบระบบต่างๆ เนื้อเรื่อง แผนที่ ปริศนาต่าง ตัวละครที่จะใช้ เป็นต้น}
    \planitem{10}{2023}{11}{2023}{ระบบควบคุม เช่น การเดิน การกระโดด มุมกล้อง เป็นต้น}
    \planitem{11}{2023}{12}{2023}{ระบบต่อสู้ เสียง การแสดงท่าทางเมื่อโจมตี}
    \planitem{11}{2023}{2}{2024}{สร้างmap รวมถึงจัดสถานที่บอส ปริศนาต่างๆ และว่างเนื้อเรื่อง}
    \planitem{12}{2023}{1}{2024}{effect และ UI ต่างๆในเกม เช่น หลอดเลือด level เป็นต้น}
    \planitem{1}{2024}{2}{2024}{รวมทุกอย่างเข้าด้วยกันให้เกมสามารถเล่นจนจบเกมได้}
    \planitem{1}{2024}{2}{2024}{รวบรวมข้อมูล ความสนใจของผู้เล่น และทำเล่มโครงงาน}
    \planitem{2}{2024}{3}{2024}{ตรวจสอบความถูกต้องของโครงงาน รวมถึงบัคต่างๆในเกม และตกแต่งเกมให้มีความสมบูรณืมากยิ่งขึ้น}
\end{plan}

\section{\ifenglish Roles and responsibilities\else บทบาทและความรับผิดชอบ\fi}
สำหรับการแบ่งงานของกลุ่มของพวกเราจะแบ่งออกเป็น 3 ส่วนหลักๆ

1.การออกแบบเกมโดยรวม: ในส่วนนี้จะช่วยกันทำ โดยการระดมความคิด ข้อเสนอต่างๆ มารวมกันเเล้วเลือกเอาในสิ่งที่ สามารถทำได้และ สิ่งเห็นตรงกันว่าอยากจะให้มีในเกมของพวกเรา

2.การออกแบบต่างๆ: สำหรับการออกเเบบส่วนใหญ่รับผิดชอบโดย นายเทวฤทธิ์ สมฤทธิ์ ไม่ว่าจะเป็น แมพ ปริศนาต่างๆ effect เสียงต่าง แต่โดยรวมเเล้วช่วยๆกันทำ โดยเฉพาะการเลือกตัวละคร และเนื้อเรื่องของเกม

3.การออกแบบระบบเกม: สำหรับการออกเเบบระบบเกมรับผิดชอบโดย นายชาญชัย ไชยสลี ไม่ว่าจะเป็นระบบควบคุมตัวละคร การออกท่าโจมตี ระบบเลือด แต่โดยรวมเเล้วช่วยๆกันทำ โดยเฉพาะระบบกาควบคุมตัวละคร ระบบการอัพเลเวล การต่อสู้

\section{\ifenglish%
Impacts of this project on society, health, safety, legal, and cultural issues
\else%
ผลกระทบด้านสังคม สุขภาพ ความปลอดภัย กฎหมาย และวัฒนธรรม
\fi}
\enskip \enskip \enskip \enskip \enskip สำหรับเกม Miracle from sky เป็นเกมที่เน้นสร้างความสนุกสนาน ความบันเทิงให้กับผู้เล่น ดังนั้นภายในเกมไม่มีฉากที่ล่อแหลม ทั้งทางเพศ และทางการกระทำผิดกฎหมาย และเนื่องจากเกมของพวกเราเน้นให้ผู้เล่นผ่อนคลาย ไม่ว่าจะเด็กหรือผู้ใหญ่ ดังนั้นในเกมจึงไม่มีการใส่ effect เลือดสาดต่างๆเข้าไป แต่ถึงอย่างนั้น 
ในเกมก็ยังมาฉากการตายของตัวละครหลัก และมอนสเตอร์ รวมถึงมีการต่อสู้ มีใช้อาวุธต่างๆ เพื่อใช้ในการฆ่าศัตรู

\enskip \enskip เกม Miracle from sky สามารถนำไปเป็นต้นแบบ เเนวทาง หรือแรงบันดาลใจ ในการทำเกมแนว action RPG OpenWorld ได้ นอกจากนั้น โครงงานนี้ยังสามารถนำไปประยุกต์ใช้งานร่วมกับโครงงานอื่นได้ เช่น โครงงานที่ศึกษาเกี่ยวกับการลดความเครียดโดยการเล่นเกม โครงงานที่ศึกษาเกี่ยวกับการเล่นเกมว่าจะส่งผลกระทบอะไรกับเรียนรู้ของเด็กบ้าง เป็นต้น


% แนวทางและโยชน์ในการประยุกต์ใช้งานโครงงานกับงานในด้านอื่นๆ รวมถึงผลกระทบในด้านสังคมและสิ่งแวดล้อมจากการใช้ความรู้ทางวิศวกรรมที่ได้
