\chapter{\ifenglish Introduction\else บทนำ\fi}

\section{\ifenglish Project rationale\else ที่มาของโครงงาน\fi}
% \tolerance = 9999
% \overfullrule=0pt
\enskip \enskip \enskip \enskip \enskip 
เกมเป็นสื่อบันเทิงประเภทหนึ่งที่มีการแพร่หลายเป็นอย่างมากในปัจจุบัน ไม่ว่าจะเป็นเกมบนมือถือ บนเว็บไซต์ เกมบนเครื่องเล่นเกมต่างๆที่ออกแบบมาเพื่อเกมใดเกมหนึ่งโดยเฉพาะ และรวมไปถึงเกมบนคอมพิวเตอร์ ซึ่งบางเกมได้มีการจัดการแข่งขันกันขึ้น เพื่อชิงรางวัลต่างๆมากมายภายในงานเเข่ง ส่งผลให้ผู้คนให้ความสนใจกับเกมมากขึ้น และส่งผลให้อุสาหกรรมเกมมีการเติบโตอย่างรวดเร็ว จนเกิดอาชีพใหม่ๆมากมายที่เกี่ยวกับเกม เช่น Streamer, นักกีฬา E-sport, นักพากย์เกม เป็นต้น   

\enskip \enskip
โดยโครงงานนี้ก็ได้เริ่มต้นมาจากการที่ผู้พัฒนาชื่นชอบในการเล่นเกม และมีความสนใจที่จะสร้างเกมขึ้นมาหนึ่งเกม ซึ่งผู้พัฒนาได้ลองทำการศึกษาพื้นฐานต่างๆเกี่ยวกับการสร้างเกม เเละตัดสินใจเสนอความสนใจเหล่านี้พร้อมกับอธิบายเหตุผลให้กับอาจารย์ฟัง จนสุดท้ายได้ทำการตกลงกับอาจารย์ว่าจะสร้างเกม 3D แนว RPG action ขึ้นมา ซึ่งก็คือ เกม Miracle from sky นั่นเอง 

\enskip \enskip สำหรับเกม Miracle from sky เป็นเกมเเนว Action RPG OpenWorld แบบ Single-player ที่มีมุมมองเป็น มุมมองของบุคคลที่สาม ซึ่งผู้พัฒนาให้ความสนใจ และอยากนำมาเป็นต้นแบบในการทำเกมคือ Genshin impact และ Diablo ซึ่งทางระบบ gameplay จะเน้นไปทาง Genshin impact ส่วนระบบสกิลจะเน้นไปทาง Diablo ซึ่งในเกม ผู้เล่นจะได้รับบทเป็นเด็กสาวที่ต้องผจญภัยในโลกกว้าง และฝึกฝนตัวเองให้เก่งขึ้น เพื่อที่จะไปปราบจอมมาร โดยระหว่างการเดินทางผู้เล่นจะได้พบศัตรูหลากหลายรูปแบบ ซึ่งต้องใช้วิธีรับมือที่แตกต่างกัน ได้สำรวจโลกแฟนตาซีกว้างใหญ่ และได้พบเจอกับปริศนาต่างๆที่รอให้ผู้เล่นได้เข้าไปแก้ไขหาคำตอบ 
\section{\ifenglish Objectives\else วัตถุประสงค์ของโครงงาน\fi}
\begin{enumerate}
    \item เพื่อตอบสนองความสนใจ ความต้องการของผู้พัฒนาที่อยากจะทำเกมของตัวเองขึ้นมาซักหนึ่งเกม
    \item เพื่อสร้างประสบการณ์ต่างๆที่น่าตื่นเต้น สนุกสนาน และน่าติดตามให้กับผู้เล่น ผ่านทางตัวเกม ทั้งด้านเนื้อเรื่อง gameplay และสิ่งต่างๆภายในเกม
    \item เพื่อสร้างความบันเทิงให้กับผู้เล่น และช่วยทำให้ผู้เล่นรู้สึกผ่อนคลายเวลาเล่นเกม
    \item เพื่อเป็นแบบอย่าง และแรงบรรดาลใจให้กับหลายๆคนที่อยากจะลองสร้างเกมของตัวเองขึ้นมา
\end{enumerate}

\section{\ifenglish Project scope\else ขอบเขตของโครงงาน\fi}

\subsection{\ifenglish Hardware scope\else ขอบเขตด้านฮาร์ดแวร์\fi}
\begin{enumerate}
    \item เกมสามารถเล่นได้ผ่านทางคอมพิวเตอร์ ซึ่งได้แก่ PC, laptop
    \item เกมจะใช้เมาส์ แป้นพิมพ์ ในการควบคุม
\end{enumerate}
\subsection{\ifenglish Software scope\else ขอบเขตด้านซอฟต์แวร์\fi}
\begin{enumerate}
    \item เกมจะรองรับแค่ระบบปฏิบัติการ Windows
    \item เกมถูกออกแบบมาสำหรับผู้เล่นคนเดียว
    \item เกมมีมุมมองเป็นเเบบมุมมองบุคคลที่สาม เท่านั้น ไม่สามารถเปลี่ยนมุมมองอื่นๆได้
\end{enumerate}
\section{\ifenglish Expected outcomes\else ประโยชน์ที่ได้รับ\fi}
\begin{enumerate}
    \item ผู้เล่นจะได้รับความสนุกสนาน ความบันเทิงต่างๆภายในเกม
    \item ผู้เล่นจะได้รับประสบการณ์ใหม่ๆมากมายจากการเกม
    \item ผู้พัฒนาได้รับประสบการณ์ใหม่ๆในการทำงานเป็นทีม และประสบการณ์ต่างๆในการสร้างเกม ซึ่งเป็นผลดีต่อการทำงานในอนาคต
\end{enumerate}
\section{\ifenglish Technology and tools\else เทคโนโลยีและเครื่องมือที่ใช้\fi}

\subsection{\ifenglish Hardware technology\else เทคโนโลยีด้านฮาร์ดแวร์\fi}
\begin{enumerate}
    \item คอมพิวเตอร์รุ่น Lenovo legion y540, core i5  ใช้ในการออกแบบ และพัฒนา
    \item คอมพิวเตอร์รุ่น Asus Rog, core i7 ใช้ในการออกแบบ และพัฒนา
\end{enumerate}
\subsection{\ifenglish Software technology\else เทคโนโลยีด้านซอฟต์แวร์\fi}
\begin{enumerate}
    \item Github ใช้ในการ พํฒนาและอัพเดต source code 
    \item Figma  ใช้ในการออกแบบหน้าตาของเว็บไซต์
    % \item Visual Studio โดยใช้ในการเขียน ในการควบระบบของเกม ซึ่งใช้ภาษา ในการเขียน code
    \item Visual studio ใช้ในการเขียน code
    \item Docker ใช้ในการจำลองการสร้างserver 
    \item Javascript ภาษาที่ใช้ในการเขียน chaincode 
    \item Photoshop ใช้ในการออกแบบ
    \item React ใช้สำหรับสร้าง user interface
\end{enumerate}
\section{\ifenglish Project plan\else แผนการดำเนินงาน\fi}

\begin{plan}{10}{2023}{3}{2024}
    \planitem{10}{2023}{11}{2023}{ศึกษาค้นคว้าการใช้งาน hyperledger fabric ฺBlockchain }
    \planitem{10}{2023}{11}{2023}{วางแผนออกแบบระบบต่างๆ}
    \planitem{11}{2023}{12}{2023}{พัฒนาฐานข้อมูลเข้าสู่ระบบ}
    \planitem{12}{2023}{1}{2024}{สร้างหน้า login registerและหน้าต่างๆ}
    \planitem{12}{2023}{1}{2024}{ทดลองtestระบบnetwork}
    \planitem{12}{2023}{1}{2024}{เขียน chaincode}
    \planitem{12}{2023}{3}{2024}{เขียน api ต่างๆ}
    \planitem{3}{2024}{4}{2024}{ทดสอบระบบและแก้ไข}
\end{plan}

\section{\ifenglish Roles and responsibilities\else บทบาทและความรับผิดชอบ\fi}


1.การออกแบบโดยรวม: ในส่วนนี้จะช่วยกันทำ โดยการระดมความคิด ข้อเสนอต่างๆ มารวมกันเเล้วเลือกเอาในสิ่งที่ สามารถทำได้และ สิ่งเห็นตรงกันว่าอยากจะให้มี

\section{\ifenglish%
Impacts of this project on society, health, safety, legal, and cultural issues
\else%
ผลกระทบด้านสังคม สุขภาพ ความปลอดภัย กฎหมาย และวัฒนธรรม
\fi}
\enskip \enskip \enskip \enskip \enskip 
โครงงานของเราได้ใช้ระบบprivate blockchain ที่มีความปลอดภัยสูงและมีความน่าเชื่อถือมากทำให้ระบบของเราสามารถตรวจสอบป้องกันการปลอมแปลงเอกสารซึ่งเป็นการกระทำที่ผิดกฎหมายแเช่วยทำให้การอยู่ร่วมกันในสังคมน่าอยู่และยังมีความปลอดภัยทางด้านข้อมูลสูงไม่มีการเปิดเผยข้อมูลที่อ่อนไหวเกินไปและสามารถรักษาความเป็นส่วนตัวของผู้ใช้งานได้ทำให้ผู้ใช้งานได้ใช้เว็บไซต์ของเราได้โดยไม่มีความกังวลใดๆ





% แนวทางและโยชน์ในการประยุกต์ใช้งานโครงงานกับงานในด้านอื่นๆ รวมถึงผลกระทบในด้านสังคมและสิ่งแวดล้อมจากการใช้ความรู้ทางวิศวกรรมที่ได้
