\chapter{\ifenglish Introduction\else บทนำ\fi}

\section{\ifenglish Project rationale\else ที่มาของโครงงาน\fi}
% \tolerance = 9999
% \overfullrule=0pt
\enskip \enskip \enskip \enskip \enskip 
ในปัจจุบันพบว่า มีซอฟต์แวร์ในอินเตอร์เน็ตที่สามารถปลอมแปลงใบประกาศนียบัตรได้ทำให้เราไม่
สามารถรู้ได้เลยว่า ประกาศนียบัตรที่เราได้มานั้นเป็นเอกสารจริงหรือปลอม ถึงแม้ว่าจะตรวจสอบได้แต่อาจจะ
ใช้เวลาค่อนข้างนานพอสมควร และเรายังไม่สามารถรู้ได้เลยว่ามีใครมาแก้ไขปลอมแปลงประกาศนียบัตรนั้น
หรือไม่จึงเกิดเป็นปัญหาที่เราไม่สามารถตรวจสอบความน่าเชื่อถือของเอกสารนั้นได้ดีพอ พวกเราได้เล็งเห็นถึง
ปัญหาในข้อนี้จึงเกิดความคิดริเริ่มในการทำเว็บไซต์นี้ขึ้นมา เพื่อลดความผิดพลาดและป้องกันการปลอมแปลง
ใบประกาศนียบัตร การใช้ Blockchain ในระบบประกาศนียบัตรอิเล็กทรอนิกส์สามารถช่วยเพิ่มความเชื่อถือ
ในเอกสาร โดยข้อมูลใบประกาศนียบัตรทั้งหมดจะถูกบันทึกลงใน Blockchain โดยที่ทุกๆการแก้ไขจะสามารถ
ตรวจสอบได้เสมอ และทำให้สามารถลดความเสี่ยงในการถูกแอบอ้างโดยเอกสารปลอม การลดความผิดพลาด
ในข้อมูลเหล่านี้ยังสามารถช่วยลดค่าใช้จ่ายในการตรวจสอบและยืนยันเอกสารได้อีกด้วย ทำให้ระบบของเรามี
ความสำคัญอย่างมากในการปรับปรุงกระบวนการทำงานและเพิ่มประสิทธิภาพในหลายๆ ด้านขององค์กรและ
สังคมอย่างรวดเร็วและปลอดภัยมากยิ่งขึ้น




\section{\ifenglish Objectives\else วัตถุประสงค์ของโครงงาน\fi}
\begin{enumerate}
    \item เพื่อความปลอดภัยของข้อมูลใบประกาศนียบัตร
    \item เพื่อแสดงความน่าเชื่อถือในประกาศนียบัตรนั้น
\end{enumerate}

\section{\ifenglish Project scope\else ขอบเขตของโครงงาน\fi}

\subsection{\ifenglish Hardware scope\else ขอบเขตด้านฮาร์ดแวร์\fi}
\begin{enumerate}
    \item คอมพิวเตอร์เพื่อให้สามารถทํางานหรือใช้เครื่องมือที่เกี่ยวข้องกับซอฟต์แวร์อื่นๆ
    \item เซิร์ฟเวอร์ การเข้าใช้งานเว็บไซต์ และ เซิร์ฟเวอร์ของฐานข้อมูลที่จัดเก็บฐานข้อมูลธรรมดาและแบบblockchain
\end{enumerate}
\subsection{\ifenglish Software scope\else ขอบเขตด้านซอฟต์แวร์\fi}
\begin{enumerate}
    \item เซิร์ฟเวอร์ ในการเข้าใช้งานเว็บไซต์ และ เซิร์ฟเวอร์ของฐานข้อมูล
    \item สร้างใบประกาศนียบัตรเก็บไว้ใน Blockchain ในรูปแบบฟอร์มผ่านเว็บไซต์
    \item เครื่องมือทดสอบการใช้งานแบบหลายเครื่อง docker
    \item blockchain software fabric hyperledger
\end{enumerate}
\section{\ifenglish Expected outcomes\else ประโยชน์ที่ได้รับ\fi}
\begin{enumerate}
    \item เพิ่มความปลอดภัยให้กับข้อมูล: ข้อมูลของเราจะเก็บไว้ใน blockchain
    \item ข้อมูลของเราทีความน่าเชื่อถือมากขึ้น: ข้อมูลของเราจะไม่สามารถแก้ไขได้โดยง่ายและ
    จะเก็บ transaction และ timestamp ไว้ทุกครั้ง
    \item เพิ่มความรวดเร็วในการตรวจสอบ สามารถตรวจสอบได้ในเว็บไซต์ของเราที่รวดเร็วยิ่งขึ้น
\end{enumerate}
\section{\ifenglish Technology and tools\else เทคโนโลยีและเครื่องมือที่ใช้\fi}

% \subsection{\ifenglish Hardware technology\else เทคโนโลยีด้านฮาร์ดแวร์\fi}
% \begin{enumerate}
%     \item คอมพิวเตอร์รุ่น Lenovo legion y540, core i5  ใช้ในการออกแบบ และพัฒนา
%     \item คอมพิวเตอร์รุ่น Asus Rog, core i7 ใช้ในการออกแบบ และพัฒนา
% \end{enumerate}
\subsection{\ifenglish Software technology\else เทคโนโลยีด้านซอฟต์แวร์\fi}
\begin{enumerate}
    \item Github ใช้ในการ พํฒนาและอัพเดต source code 
    \item Figma  ใช้ในการออกแบบหน้าตาของเว็บไซต์
    % \item Visual Studio โดยใช้ในการเขียน ในการควบระบบของเกม ซึ่งใช้ภาษา ในการเขียน code
    \item Visual studio ใช้ในการเขียน code
    \item Docker ใช้ในการจำลองการสร้างserver 
    \item golang ภาษาที่ใช้ในการเขียน chaincode 
    \item Photoshop ใช้ในการออกแบบ
    \item React ใช้สำหรับสร้าง user interface
\end{enumerate}
\section{\ifenglish Project plan\else แผนการดำเนินงาน\fi}

\begin{plan}{10}{2023}{3}{2024}
    \planitem{10}{2023}{11}{2023}{ศึกษาค้นคว้าการใช้งาน hyperledger fabric ฺBlockchain }
    \planitem{10}{2023}{11}{2023}{วางแผนออกแบบระบบต่างๆ}
    \planitem{11}{2023}{12}{2023}{พัฒนาฐานข้อมูลเข้าสู่ระบบ}
    \planitem{12}{2023}{1}{2024}{สร้างหน้า login registerและหน้าต่างๆ}
    \planitem{12}{2023}{1}{2024}{ทดลองtestระบบnetwork}
    \planitem{12}{2023}{1}{2024}{เขียน chaincode}
    \planitem{12}{2023}{3}{2024}{เขียน api ต่างๆ}
    \planitem{3}{2024}{4}{2024}{ทดสอบระบบและแก้ไข}
\end{plan}


\section{\ifenglish Roles and responsibilities\else บทบาทและความรับผิดชอบ\fi}


1.การออกแบบโดยรวม: ในส่วนนี้จะช่วยกันทำ โดยการระดมความคิด ข้อเสนอต่างๆ มารวมกันเเล้วเลือกเอาในสิ่งที่ สามารถทำได้และ สิ่งเห็นตรงกันว่าอยากจะให้มี
2.back-end: โดยส่วนนี้นายคนธกานต์ และ คุณาสิน จะช่วยกัน
3.front-end: คุณาสินจะเป็นคนทำ
4.รายงาน: คนธกานต์จะเป็นคนทำแต่ขอข้อมูลบ้างอย่างมาจากคุณาสิน

\section{\ifenglish%
Impacts of this project on society, health, safety, legal, and cultural issues
\else%
ผลกระทบด้านสังคม สุขภาพ ความปลอดภัย กฎหมาย และวัฒนธรรม
\fi}
\enskip \enskip \enskip \enskip \enskip 
โครงงานของเราได้ใช้ระบบprivate blockchain ที่มีความปลอดภัยสูงและมีความน่าเชื่อถือมากทำให้ระบบของเราสามารถตรวจสอบป้องกันการปลอมแปลงเอกสารซึ่งเป็นการกระทำที่ผิดกฎหมายแช่วยทำให้การอยู่ร่วมกันในสังคมน่าอยู่และยังมีความปลอดภัยทางด้านข้อมูลสูงไม่มีการเปิดเผยข้อมูลที่อ่อนไหวเกินไปและสามารถรักษาความเป็นส่วนตัวของผู้ใช้งานได้ทำให้ผู้ใช้งานได้ใช้เว็บไซต์ของเราได้โดยไม่มีความกังวลใดๆ





% แนวทางและโยชน์ในการประยุกต์ใช้งานโครงงานกับงานในด้านอื่นๆ รวมถึงผลกระทบในด้านสังคมและสิ่งแวดล้อมจากการใช้ความรู้ทางวิศวกรรมที่ได้
