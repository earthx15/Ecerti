\chapter{\ifenglish Background Knowledge and Theory\else ทฤษฎีที่เกี่ยวข้อง\fi}
\tolerance = 9999
\overfullrule=0pt

% การทำโครงงาน เริ่มต้นด้วยการศึกษาค้นคว้า ทฤษฎีที่เกี่ยวข้อง หรือ งานวิจัย/โครงงาน ที่เคยมีผู้นำเสนอไว้แล้ว ซึ่งเนื้อหาในบทนี้ก็จะเกี่ยวกับการอธิบายถึงสิ่งที่เกี่ยวข้องกับโครงงาน เพื่อให้ผู้อ่านเข้าใจเนื้อหาในบทถัดๆ ไปได้ง่ายขึ้น
\enskip \enskip \enskip \enskip \enskip Miracle from sky ถูกสร้าง และพัฒนาขึ้นโดยโปรแกรม Unity เป็นหลัก
 ซึ่งก่อนที่ผู้พัฒนาจะเริ่มลงมือสร้างเกมจริงขึ้นมา ผู้พัฒนาได้ศึกษาหาความารู้ในด้านต่างๆที่จำเป็นสำหรับการสร้างเกม  
โดยเนื้อหาในบทนี้จะอธิบายในส่วนของความรู้ ทฤษฎีบทที่เกี่ยวข้อง และหลักการต่างๆที่ผู้พัฒนาได้ศึกษา และนำไปใช้ในการสร้างเกม 
เพื่อให้ผู้ที่เข้ามาอ่านได้เข้าใจหลักการต่างๆในเบื้องต้น และเพื่อให้เข้าใจเนื้อหาในบทถัดๆไปได้ง่ายมากยิ่งขึ้น

\section{พื้นฐาน Unity}
\enskip \enskip \enskip \enskip \enskip Unity เป็น software ที่ถูกออกแบบมาเพื่อใช้สำหรับการพัฒนา software ที่สามารถ
จำลองการทำงานต่างๆได้ เช่น game(ทั้ง 2D และ 3D), การขนส่ง, animation, อุตสาหกรรมยานยนต์ เป็นต้น 
ซึ่งสามารถลองรับได้หลากหลาย platform เช่น PC, iOS, Android เป็นต้น โดยในที่นี้จะขออธิบายในส่วนที่เกี่ยวข้องกับการสร้าง
game 3D เท่านั้น

\enskip \enskip โดยส่วนต่างๆใน Unity ที่สำคัญ และจำเป็นต้องศึกษาสำหรับการสร้างเกม มีดังนี้
\subsection{Scene}
คือ ฉากภายในเกม หรือบริเวณที่เรานำสิ่งต่างๆมาใช้รวมกัน ซึ่ง scene มีได้หลาย scene เช่น scene เริ่มเกม,
 scene จบเกม, scene เมนู เป็นต้น
\subsection{GameObject}
คือ วัตถุ หรือสิ่งต่างๆที่สามารถนำมาใช้แสดงผลภายใน scene ได้ เช่น Model ต่างๆ, ตัวเล่นเสียง,
ตัวเล่น effect, light, terrain เป็นต้น
\subsection{Asset}
คือ GameObject หรือสิ่งต่างๆที่นำเข้ามาใช้งานใน project ของเรา เช่น Model ตัวละคร, เสียง, animation,
script, texture, prefab, terrain เป็นต้น โดย asset เราสามารถซื้อจาก Unity Asset Store ได้ ซึ่งมีทั้ง
ที่แจกฟรี และเสียเงิน โดยราคาขึ้นกับคุณภาพของ asset และความพึงพอใจของผู้ขาย
\subsection{Camera}
คือ กล้องที่ใช้สำหรับการแสดงผลเกมของเราออกมาให้ผู้เล่นเห็นทางจอภาพ โดยสามารถปรับมุมมอง ตำแหน่งต่างของกล้องได้อย่างอิสระ 
สามารถตั้งให้กล้องติดตามตัวผู้เล่นได้ รวมไปถึงใช้ในการทำ cutscene
\subsection{Light}
คือ GameObject ประเภทหนึ่งที่สามารถให้แสงสว่างกับ scene ของเราได้ light ทำให้เกิดเงาของ GameObject 
ซึ่งสามารถไปปรับใช้งานได้หลากหลาย เช่น ทำเวลากลางวัน/กลางคืน, ทำ scene มืดๆที่ทำให้รู้สึกถึงความน่ากลัว เป็นต้น
\subsection{Component}
คือ คุณสมบัติ หรือความสามารถต่างๆที่อยู่ใน GameObject ซึ่งมีหลากหลายคุณสมบัติ และคุณสมบัติแต่ละตัวก็มีความแตกต่างกันไป
โดย component ที่สำคัญมีดังนี้ 

\begin{enumerate}
\item Transform คือ component ที่ใช้ในการควบคุมตำแหน่ง(Position) การหมุน(Rotation) และขนาด(Scale) 
โดยทุก GameObject ต้องมีคุณสมบัตินี้
\item Rigidbody คือ component ที่ใช้ในการจัดการเกี่ยวกับระบบฟิสิกส์ของวัตถุ ไม่ว่าจะเป็น แรง(Force), มวล(Mass), 
การแสดงผลจากแรงโน้มถ่วง(Gravity) และการล็อควัตถุ(Freeze)
\item Collider คือ component ที่ใช้ในการตรวจสอบการชนกันของวัตถุต่างๆภายใน scene นำมาประยุกต์ใช้ได้หลากหลายแบบ
เช่น การคำนวณดาเมจ, การระเบิดของลูกบอลไฟเมื่อชนกับวัตถุต่างๆ, การเก็บไอเทมต่างๆ เป็นต้น
\item Animator คือ component ที่ใช้ในการควบคุมการทำงานของ animation ต่างๆ โดยจะควบคุม และแสดงในรูปของ state machine
\item Particle System คือ component ที่ใช้ในการสร้าง visial effect หรือที่เรียกว่า VFX เช่น เปลวไฟ, สายฟ้า, น้ำ เป็นต้น
\item Volume คือ component ที่ใช้ในการควบคุมการแสดงผลทางหน้าจอ หรือภาพที่เราเห็นผ่านทางกล้อง โดยสามารถปรับบริมาณแสงที่ผ่านกล้อง, 
การเบลอขอบจอภาพ, การเพิ่มมุมแบบ perspective ทำให้ภาพที่เห็นถูกยืด หรือหดลงได้
\end{enumerate}

\subsection{Texture}
คือ รูปภาพพื้นผิวต่างๆ ที่ใช้ในการนำมาเป็นผิวของวัตถุ ช่วยให้วัตถุมีความสมจริงมากขึ้น
\subsection{Material}
คือ เม็ดสี หรือสีที่ใช้ลงสีให้กับวัตถุต่างภายใน scene ไม่จำเป็นต้องเป็นสีล้วน โดยถ้าหากใช้ shader graph
ในการสร้าง material จะทำให้ material ที่ได้มีคุณสมบัติที่กำหนดไว้ได้ เช่น material ที่เรืองแสงได้, 
material ที่มีความมันวาว, material ที่เปลี่ยนรูปร่างได้ เป็นต้น
\subsection{SkyBox}
คือ สิ่งที่ให้เปลี่ยน สี รูปแบบ คุณสมบัติต่างๆของท้องฟ้าภายในเกม เช่น สามารถทำให้ท้องฟ้าเป็นกลางคืน/วันได้,
ทำให้ท้องฟ้ามีก้อนเมฆได้ เป็นต้น
\subsection{Wind Zone}
คือ GameObject ประเภทหนึ่ง ทำหน้าที่ช่วยควบคุมการทำงานของระบบลมภายใน scene ช่วยให้เกมมีความสมจริงมากยิ่งขึ้น
\subsection{Terrain}
คือ GameObject ประเภทหนึ่ง ทำหน้าที่ช่วยควบคุม ปรับแต่งภูมิประเทศ หรือสภาพแวดล้อมของพื้น ให้มีลักษณะตามที่เราต้องการ
เช่น ใช้ทำหลุม, ใช้ทำภูเขา, ใช้ทำพื้นที่ยกระดับ เป็นต้น
\subsection{Prefab}
คือ การนำ GameObject ต่างๆมาประกอบกันเพื่อสร้างเป็น GameObject ใหม่ที่รวม GameObject หลายๆตัวเอาไว้ ซึ่งจะมีลักษณะ
ต่างๆตามที่เรากำหนด
\subsection{Tag}
คือ สิ่งที่ใช้กำหนด หรือจำแนกประเภทของ GameObject ตามที่เรากำหนด จะใช้ประโยชน์ในการตรวจสอบว่า GameObject นี้คืออะไร 
เช่น สร้าง tag ชื่อ enemy กับ tag ชื่อ player เพื่อใช้ในการระบุว่า GameObject นี้คือ enemy หรือ player เป็นต้น 
\subsection{Layer}
คือ ลำดับชั้นการแสดงผล รวมถึงการทำงานของวัตถุ โดย layer สูงๆ หรือ layer ที่มีเลขต่ำๆ จะมีสิทธิ์ถูกสั่งให้แสดงผล 
หรือ ได้ทำงานก่อนเป็นลำดับแรกๆ
\subsection{Script}
คือ ส่วนของ code ที่ใช้ในการควบคุมการทำงานต่างๆภายในเกม ตั้งเริ่มเกม จนจบเกม โดยในส่วน script ที่ใช้ในใน Unity
จะถูกเขียนโดยภาษา C$\sharp$ เป็นหลัก ซึ่ง script จะถูกใช้งานได้โดยการรับ input ต่างๆจาก GameObject เช่น 
การกด w, a, s, d ในการสั่งให้ GameObject เคลื่อนที่ไปในทิศทางต่างๆ, การกด spacebar ในการสั่ง GameObject กระโดด เป็นต้น

\section{พื้นฐาน Blender}
\enskip \enskip \enskip \enskip \enskip Blender เป็น software ที่ใช้สำหรับสร้างงานทางสาย graphic 3D สามารถสร้าง Model 3D 
ได้ ทำ texture ได้ รวมถึงสามารถทำ animation ได้ โดย Blender รองรับได้หลากหลายระบบปฏิบัติการไม่ว่าจะเป็น Windows, Mac OS, Linux แนวทางและโยชน์ในการประยุกต์ใช้งานโครงงานกับงานในด้านอื่นๆ

\enskip \enskip โดยส่วนต่างๆใน Blender ที่สำคัญ และช่วยในการสร้างเกม มีดังนี้
\subsection{Workspaces}
คือ หน้าต่างการทำงานต่างๆในโปรแกรม Blender ซึ่งในแต่ละหน้าจะทำหน้าที่แตกต่างกันไป โดยหน้าสำคัญๆมีดังนี้
\begin{enumerate}
\item Layout คือ หน้าหลักที่ใช้สำหรับออกแบบ และปั้น Model ต่างๆ
\item UV Editing คือ หน้าที่ใช้สำหรับการทำ UV texture ซึ่งใช้สำหรับการลงสี หรือพื้นผิวของ Model ซึ่งจะแสดงในรูปแบบภาพคลี่ของ Model
\item Shading คือ หน้าที่ใช้สำหรับควบคุมลักษณะของผิว Model เช่น ให้มีความมันวาว, มีความด้าน, ให้มีสีที่แตกต่าง เป็นต้น
\item Animation คือ หน้าที่ใช้สำหรับการออกแบบ สร้าง animation ต่างๆให้กับ Model เช่น ท่าทางการขยับตัว, การกระโดด, การกระพริบตา เป็นต้น
\end{enumerate}
\subsection{Mode}
ใน Blender จะแบ่งการทำงานออกเป็น mode ซึ่งการทำงานของแต่ละ mode จุดประสงค์ และการทำงานที่แตกต่างกันออกไป โดยมี mode ที่สำคัญๆดังนี้
\begin{enumerate}
\item Object Mode คือ mode หลักที่ใช้สำหรับออกแบบ และปั้น Model ต่างๆ โดย mode นี้จะสร้าง object ต่างๆขึ้นมาได้ เช่น 
ทรงกลม, ทรงกระบอก, แผ่นระนาบ เป็นต้น
\item Weight Paint คือ mode ที่ใช้ในการควบคุม จัดการกับน้ำหนักของ Model โดยสามารถกำหนดน้ำส่วนต่างๆของ Model ได้ตามที่ต้องการ 
ซึ่ง จะใช้ประโยชน์ตอนทำ animtion จะช่วยให้ animation ดูสมจริงมากยิ่งขึ้น
\item Texture Paint คือ mode ที่ใช้ในการลงสี texture โดยจะสามารถลงสีได้โดยตรงที่ตัว Model เลย
\item Edit Mode คือ mode ที่ใช้ในการจัดการกับ vertex, edge, face ของ Model ใช้สำหรับการต่อ-การดึง Model เช่น 
ใช้ในการสร้างแขน-ขาของ Model, ใช้สร้างของประดับตัว Model เป็นต้น
\item Sculpt Mode คือ mode ที่ใช้ในการปั้น Model ไม่ว่าจะเป็นการยืด การหด การทำให้ผิวเรียบเนียน
\end{enumerate}

\section{การเขียนโปรแกรมเชิงวัตถุ(Object Oriented Programming : OOP)}
\enskip \enskip \enskip \enskip \enskip การเขียนโปรแกรมเชิงวัตถุ เป็นการเขียนโปรแกรมประเภทหนึ่ง โดยใช้แนวคิดในการพัฒนา software 
ที่มอง code เป็นวัตถุ(object)แทนการเขียน code เป็น streaming ต่อกันยาวๆ การเขียนโปรแกรมเชิงวัตถุ
ช่วยทำให้ Developer เห็นภาพรวมของ Code ได้ง่ายขึ้น สามารถทำความเข้าใจ และแก้ไขข้อผิดพลาดต่างๆได้ถูกจุดอย่างรวดเร็ว 
เพราะการเขียน Code แบบโปรแกรมเชิงวัตถุ code จะถูกแบ่งเป็นส่วนๆ(class)อย่างชัดเจน ซึ่งช่วยให้หา Code ได้ง่ายขึ้น  
นอกจากนั้น หลักการการเขียนโปรแกรมเชิงวัตถุไม่ได้ยึดติดกับภาษาในการเขียนภาษาใดภาษาหนึ่ง ดังนั้นการเขียนโปรแกรมเชิงวัตถุ หรือ OOP จึงออกแบบมาเพื่อให้ code
ที่เราเขียนมีแบบแผน เหมาะสมในการพัฒนา software ที่ซับซ้อน ซื่งในการสร้างเกมก็ต้องใช้หลัการของ OOP ในการเขียน code เพื่อใช้ในการควบคุมการทำงานต่างๆภายในเกม 
หรือที่เรียกว่า script

\enskip \enskip \enskip \enskip \enskip ในการเขียนโปรแกรมเชิงวัตถุ เราจะเทียบ code กับวัตถุในชีวิตจริง เช่น มนุษย์(player) ซึ่งจะให้ player เป็นวัตถุ 
โดยสิ่งที่ player ต้องมีก็คือ คุณสมบัติ(Attribute) เช่น ผู้ชาย, สูง, ผมสั้น เป็นต้น และต้องมี พฤติกรรม(Behavior) เช่น เดิน, วิ่ง, กระโดด เป็นต้น 
การเขียนโปรแกรมก็ต้องทำให้ code ของเรามีคุณสมบัติและพฤติกรรมเช่นเดียวกันกับ player ซึ่งทำได้โดยการสร้าง class ของ player ขึ้นมา โดยใน class ที่สร้าง
นั้นจะมีทั้ง Attribute และ Behavior ของ player ดังที่กล่าวมา และเมื่อจะนำ class ไปใช้งาน จะทำได้โดยการสร้าง object ขึ้นมา เปรียบเสมือนเป็น player หนึ่งคน
โดย player ที่สร้างมานั้น จะมีทั้ง Attribute และ Behavior เหมือนของ class player ดังกล่าว


\enskip \enskip \enskip \enskip \enskip สำหรับการเขียนโปรแกรมเชิงวัตถุ หรือ OPP ประกอบไปด้วยหลักการที่สำคัญอยู่ 4 ข้อ ได้แก่ 
Encapsulation Abstraction Inheritance Polymorphism
\begin{enumerate}
\item Layout คือ หน้าหลักที่ใช้สำหรับออกแบบ และปั้น Model ต่างๆ
\end{enumerate}
\subsubsection{Subsubsection 1 heading goes here}
Subsubsection 1 text

\subsubsection{Subsubsection 2 heading goes here}
Subsubsection 2 text

\section{Third section}
Section 3 text. The dielectric constant\index{dielectric constant}
at the air-metal interface determines
the resonance shift\index{resonance shift} as absorption or capture occurs
is shown in Equation~\eqref{eq:dielectric}:

\begin{equation}\label{eq:dielectric}
k_1=\frac{\omega}{c({1/\varepsilon_m + 1/\varepsilon_i})^{1/2}}=k_2=\frac{\omega
\sin(\theta)\varepsilon_\mathit{air}^{1/2}}{c}
\end{equation}

\noindent
where $\omega$ is the frequency of the plasmon, $c$ is the speed of
light, $\varepsilon_m$ is the dielectric constant of the metal,
$\varepsilon_i$ is the dielectric constant of neighboring insulator,
and $\varepsilon_\mathit{air}$ is the dielectric constant of air.

\section{About using figures in your report}

% define a command that produces some filler text, the lorem ipsum.
\newcommand{\loremipsum}{
  \textit{Lorem ipsum dolor sit amet, consectetur adipisicing elit, sed do
  eiusmod tempor incididunt ut labore et dolore magna aliqua. Ut enim ad
  minim veniam, quis nostrud exercitation ullamco laboris nisi ut
  aliquip ex ea commodo consequat. Duis aute irure dolor in
  reprehenderit in voluptate velit esse cillum dolore eu fugiat nulla
  pariatur. Excepteur sint occaecat cupidatat non proident, sunt in
  culpa qui officia deserunt mollit anim id est laborum.}\par}

\begin{figure}
  \centering

  \fbox{
     \parbox{.6\textwidth}{\loremipsum}
  }

  % To include an image in the figure, say myimage.pdf, you could use
  % the following code. Look up the documentation for the package
  % graphicx for more information.
  % \includegraphics[width=\textwidth]{myimage}

  \caption[Sample figure]{This figure is a sample containing \gls{lorem ipsum},
  showing you how you can include figures and glossary in your report.
  You can specify a shorter caption that will appear in the List of Figures.}
  \label{fig:sample-figure}
\end{figure}

Using \verb.\label. and \verb.\ref. commands allows us to refer to
figures easily. If we can refer to Figures
\ref{fig:walrus} and \ref{fig:sample-figure} by name in the {\LaTeX}
source code, then we will not need to update the code that refers to it
even if the placement or ordering of the figures changes.

\loremipsum\loremipsum

% This code demonstrates how to get a landscape table or figure. It
% uses the package lscape to turn everything but the page number into
% landscape orientation. Everything should be included within an
% \afterpage{ .... } to avoid causing a page break too early.
\afterpage{
  \begin{landscape}
  \begin{table}
    \caption{Sample landscape table}
    \label{tab:sample-table}

    \centering

    \begin{tabular}{c||c|c}
        Year & A & B \\
        \hline\hline
        1989 & 12 & 23 \\
        1990 & 4 & 9 \\
        1991 & 3 & 6 \\
    \end{tabular}
  \end{table}
  \end{landscape}
}

\loremipsum\loremipsum\loremipsum

\section{Overfull hbox}

When the \verb.semifinal. option is passed to the \verb.cpecmu. document class,
any line that is longer than the line width, i.e., an overfull hbox, will be
highlighted with a black solid rule:
\begin{center}
\begin{minipage}{2em}
juxtaposition
\end{minipage}
\end{center}

\section{\ifenglish%
\ifcpe CPE \else ISNE \fi knowledge used, applied, or integrated in this project
\else%
ความรู้ตามหลักสูตรซึ่งถูกนำมาใช้หรือบูรณาการในโครงงาน
\fi
}

อธิบายถึงความรู้ และแนวทางการนำความรู้ต่างๆ ที่ได้เรียนตามหลักสูตร ซึ่งถูกนำมาใช้ในโครงงาน

\section{\ifenglish%
Extracurricular knowledge used, applied, or integrated in this project
\else%
ความรู้นอกหลักสูตรซึ่งถูกนำมาใช้หรือบูรณาการในโครงงาน
\fi
}

อธิบายถึงความรู้ต่างๆ ที่เรียนรู้ด้วยตนเอง และแนวทางการนำความรู้เหล่านั้นมาใช้ในโครงงาน
