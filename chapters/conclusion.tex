\chapter{\ifenglish Conclusions and Discussions\else บทสรุปและข้อเสนอแนะ\fi}

\section{\ifenglish Conclusions\else สรุปผล\fi}
\enskip \enskip \enskip \enskip \enskip
ในการทำโครงงานนี้จัดทำเพื่อป้องกันการปลอมแปลงใบประกาศนียบัตรโดยใช้เทคโนโลยี Blockchain ได้ผลลัพธ์ที่มีความสําคัญและเป็นประโยชน์ในหลายๆ ด้าน ทำให้ประกาศนียบัตรมีน่าเชื่อถือมากขึ้นเพราะข้อมูลไม่สามารถปลอมแปลงหรือแก้ไขได้โดยไม่ได้รับอนุญาตและถ้าโดนแก้ไข้จะสามารถรู้ได้ทันทีว่าถูกแก้ตรงไหน ซึ่งสามารถช่วยลดความเสี่ยงที่เอกสารจะถูกแอบอ้างโดยการใช้เอกสารปลอมได้ นอกจากนี้ยังช่วยเพิ่มความน่าเชื่อถือในการตรวจสอบเอกสารของบุคคลภายนอกด้วยการให้ข้อมูลที่สามารถตรวจสอบได้
และยังช่วยในการลดค่าใช้จ่ายในการตรวจสอบและยืนยันเอกสารและความปลอดภัยของข้อมูลที่มีการเข้ารหัสโดยBlockchain
นอกจากนี้ยัง


\section{\ifenglish Challenges\else ปัญหาที่พบและแนวทางการแก้ไข\fi}
\enskip \enskip \enskip \enskip \enskip
ในการทำโครงงานนี้ พบว่าเกิดปัญหาหลักๆ ดังนี้
\begin{enumerate}
    \item เรื่องของ Blockchain เป็นความรู้ใหม่ที่ผู้พัฒนายังไม่เคยศึกษามาก่อนทำให้ต้องใช้เวลานานในการศึกษาเป็นอย่างมาก
    \item ข้อมูลของ Hyperledger fabric ค่อนข้างเก่าไม่มีการอัพเดตทำให้ยากต่อการศึกษา
    \item Private blockchain ไม่ได้เป็นที่นิยมขนาดนั้นเพราะยิ่งที่นิยมคือ public blockchain ซึ่งใช้ Cryptocurrency พอเริ่มหมดความนิยมทำให้ไม่ค่อยมีคนสนใจทำให้หาข้อมูลมาศึกษาได้ยาก
\end{enumerate}

\section{\ifenglish Challenges\else ข้อเสนอแนะและแนวทางการพัฒนาต่อ\fi}
\enskip \enskip \enskip \enskip \enskip
ข้อเสนอแนะเพื่อพัฒนาโครงงานนี้ต่อไป มีดังนี้
\begin{enumerate}
    \item ทำให้ระบบสามารถกำหนด Template ของใบประกาศนียบัตรได้
    \item ทำให้ระบบสามารถแสดงข้อมูลต่างๆของ Blockchain ที่ผู้ใช้ควรรู้เพื่อ
    ให้ผู้ใช้เห็นว่าข้อมูลอยู่ใน Blockchain
    \item ปรับปรุง UI ให้ดีขึ้น
    \item ทำให้ระบบประมวลผลได้รวดเร็วขึ้น
\end{enumerate}
