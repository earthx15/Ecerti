\maketitle
\makesignature

\ifproject
\begin{abstractTH}
% เขียนบทคัดย่อของโครงงานที่นี่
\enskip \enskip \enskip \enskip \enskip  ในปัจจุบัน ประกาศนียบัตร (Certificate) จากการศึกษา หรือฝึกอบรม มีแนวโน้มที่จะเป็นรูปแบบกระดาษ (Paper-based) ลดน้อยลง โดยมีการเปลี่ยนไปใช้รูปแบบอิเล็กทรอนิกส์มากขึ้นเรื่อย ๆ ปัญหาของการปลอมแปลงประกาศนียบัตรยังคงมีอยู่ และมีแนวโน้มที่จะรุนแรงมากขึ้นถึงแม้ว่าจะเป็นรูปแบบอิเล็กทรอนิกส์ก็ตาม เราไม่สามารถรู้ได้เลยว่าประกาศนียบัตรนั้นเป็นของจริงหรือของปลอม
จากการสำรวจพบว่ามีซอฟต์แวร์ในการปลอมแปลงเอกสารเหล่านี้เกิดขึ้นมาจำนวนมากในโลกออนไลน์
และพบว่าร้อยล่ะสามสิบของทั่วโลกมีการปลอมคุณสมบัติขึ้นที่ตนไม่มีขึ้นมา เราจีงต้องระวังข้อมูลปลอมและหาวิธีการยืนยันเอกสารเหล่านั้นให้ได้
 ในโครงงานนี้จึงมีแนวคิดในการออกแบบและพัฒนาระบบที่ออกใบประกาศนียบัตรอิเล็กทรอนิกส์โดยใช้ไฮเปอร์เลจเจอร์เฟบริกบล็อคเชน เพื่อเพิ่มความน่าเชื่อถือ ความถูกต้อง และความปลอดภัยในการรับรอง การยืนยัน และการตรวจสอบประกาศนียบัตร (Certificate) ซึ่งเทคโนโลยีบล็อกเชนมีคุณลักษณะพื้นฐานที่สามารถเก็บข้อมูลได้อย่างปลอดภัย มีการใช้ Cryptography ในการป้องกันการแก้ไขข้อมูลในข้อมูลที่เก็บเป็นลักษณะของบล็อกที่เชื่อมต่อกันเป็นเชน โดยโครงงานนี้ใช้เทคโนโลยีไฮเปอร์เลจเจอร์เฟบริกบล็อคเชน ที่เป็นบล็อกเชนที่ต้องได้รับอนุญาตในการเข้าร่วม (Permissioned Blockchain) มีการใช้ระบบ Digital Signature ในการยืนยันตัวตนในการเข้าถึงข้อมูลผ่าน Smart Contract (Chaincode) เพื่อความปลอดภัย และความน่าเชื่อถือของข้อมูลที่เก็บในบล็อกเชน 

% การเขียนรายงานเป็นส่วนหนึ่งของการทำโครงงานวิศวกรรมคอมพิวเตอร์
% เพื่อทบทวนทฤษฎีที่เกี่ยวข้อง อธิบายขั้นตอนวิธีแก้ปัญหาเชิงวิศวกรรม และวิเคราะห์และสรุปผลการทดลองอุปกรณ์และระบบต่างๆ
% \enskip 
% อย่างไรก็ดี การสร้างรูปเล่มรายงานให้ถูกรูปแบบนั้นเป็นขั้นตอนที่ยุ่งยาก
% แม้ว่าจะมีต้นแบบสำหรับใช้ในโปรแกรม Microsoft Word แล้วก็ตาม
% แต่นักศึกษาส่วนใหญ่ยังคงค้นพบว่าการใช้งานมีความซับซ้อน และเกิดความผิดพลาดในการจัดรูปแบบ กำหนดเลขหัวข้อ และสร้างสารบัญอยู่
% \enskip ภาควิชาวิศวกรรมคอมพิวเตอร์จึงได้จัดทำต้นแบบรูปเล่มรายงานโดยใช้ระบบจัดเตรียมเอกสาร
% \LaTeX{} เพื่อช่วยให้นักศึกษาเขียนรายงานได้อย่างสะดวกและรวดเร็วมากยิ่งขึ้น
\end{abstractTH}

\begin{abstract}
\enskip \enskip \enskip \enskip \enskip Currently, the trend for certificates from education or training programs to shift away from paper-based formats is diminishing, with an increasing reliance on electronic formats. However, the issue of certificate fraud persists, with a growing tendency towards more sophisticated methods, even with electronic formats. It's difficult to ascertain the authenticity of certificates, whether they are electronic or not.

Surveys have revealed a significant proliferation of software for document forgery in the online world. Approximately thirty percent of the global population has experienced falsification of credentials they didn't possess. Hence, there's a need to be cautious of counterfeit information and find ways to authenticate such documents.

In this project, the concept revolves around designing and developing a system for issuing electronic certificates using Hyperledger Fabric blockchain technology 
to enhance credibility, accuracy, and security in certification, verification, and validation processes. Blockchain technology offers fundamental features for securely 
storing data, utilizing cryptography to prevent data tampering within interconnected blocks forming a chain.
This project utilizes Hyperledger Fabric blockchain, which is a permissioned blockchain requiring authorization to participate. It employs Digital Signature systems for identity 
Verification to access data via Smart Contracts (Chaincode) for security and reliability of data stored in the blockchain.
% The abstract would be placed here. It usually does not exceed 350 words
% long (not counting the heading), and must not take up more than one (1) page
% (even if fewer than 350 words long).

% Make sure your abstract sits inside the \texttt{abstract} environment.
\end{abstract}

\iffalse
\begin{dedication}
This document is dedicated to all Chiang Mai University students.

Dedication page is optional.
\end{dedication}
\fi % \iffalse

\begin{acknowledgments}
\enskip \enskip \enskip \enskip \enskip โครงงานนี้จะสำเร็จลุล่วงได้ด้วยความกรุณาจากอาจารย์ รศ.ดร.ตรัสพงศ์ ไทยอุปถัมภ์     อาจารย์ที่ปรึกษาที่ได้เสียสละเวลาอันมีค่าแก่นักศึกษาโครงงานนี้
ได้รับข้อเสนอแนะและแนวคิดตลอดจนการแก้ไขข้อบกพร่อง รายละเอียดต่างๆตรวจทานแก้ไขด้วยความเอาใจใส่เป็นอย่างยิ่งจนโครงการฉบับนี้สำเร็จสมบูรณ์ลุล่สงไป ได้ด้วยดีขอกราบของพระคุณเป็นอย่างสูงไว้ณที่นี้จากใจจริงรวม\
ถึง ผศ.โดม โพธิกานนท์ และ ผศ.ดร.กำพล วรดิษฐ์ ที่ให้คําปรึกษา คำแนะนำ จนทําให้โครงงานเล่มนี้มีความสมบูรณ์มากที่สุด

\enskip \enskip ขอบคุณคณะวิศวกรรมศาสตร์ มหาวิทยาลัยเชียงใหม่ ที่ให้สถานที่ในการทําโครงงาน ทั้งห้องภาควิชาวิศวกรรมคอมพิวเตอร์ และสถานที่ต่างๆในภาควิชา และยังให้การสนับสนุนทางด้านงบประมาณ อุปกรณ์ต่างๆ ที่จำเป็นต่อการทำโครงงาน

\enskip \enskip ขอขอบพระคุณผู้ปกครอง เพื่อนๆ และรุ่นพี่ทุกๆคน ที่ให้คำปรึกษา คำแนะนำ เเละคอยเป็นกำลังใจให้ตลอดมา ซึ่งเป็นแรงผลักดันให้แก่ผู้จัดทํามีความตั้งใจและมุ่งมั่นในการทำงาน จนโครงงานที่ความสมบูรณ์มากที่สุด 

\enskip \enskip นอกจากนี้ผู้จัดทําขอขอบพระคุณอีกหลายๆท่านที่ไม่ได้กล่าวถึง ณ ที่นี้ ที่ได้ให้ความช่วยเหลือตลอดมา และสุดท้ายนี้ หากโครงงานนี้มีข้อผิดพลาดประการใด ผู้จัดทําขออภัยมา ณ ที่นี้ และพร้อมน้อมรับด้วยความยินดี

% Your acknowledgments go here. Make sure it sits inside the

% \texttt{acknowledgment} environment.

\acksign{2024}{3}{25}
\end{acknowledgments}%
\fi % \ifproject

\contentspage

\ifproject
\figurelistpage

\tablelistpage
\fi % \ifproject

% \abbrlist % this page is optional

% \symlist % this page is optional

% \preface % this section is optional
